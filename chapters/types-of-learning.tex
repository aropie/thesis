\section{Sobre inteligencia artificial}
Un factor que ha sido central en la historia del ser humano es
precisamente el que nos da el nombre como especie \sout{Homo Sapiens}:
la inteligencia. Durante años se ha buscado entender cómo es que pensamos,
es decir, la forma en que un ente puede percibir el entorno y a partir
de ello entenderlo, e incluso manipularlo y hacer predicciones al respecto.
La \textbf{inteligencia artificial} lleva este estudio un paso más adelante;
busca no sólo entender sino también construir entes inteligentes.

\subsection{La prueba de Turing}
Dado que la inteligencia artificial busca crear entes inteligentes,
es importante definir primero qué es la inteligencia. Al ser nosotros
mismos como especie nuestro modelo de inteligencia, es razonable pensar
en la inteligencia en términos de pensamiento humanos.

La \textbf{prueba de Turing}, propuesta por Alan Turing (1950), fué diseñada
para dar una definición de inteligencia en computadoras. Una computadora pasa
la prueba si, tras ser interrogada de forma escrita por un evaluador humano,
este no puede distinguir si las respuestas provienen de un humano o una computadora.
La discusión sobre si el pasar la prueba realmente es señal de una \textit{inteligencia}
sale del alcance de esta tésis, pero es importante resaltar que la prueba da
cierta noción sobre lo que se busca en una "inteligencia artificial".
Se busca que el ente "inteligente" sea capáz de tener un razonamiento racional.
Es decir, que pueda sacar conclusiones lógicas a partir de premisas dadas a través
de silogismos, como lo haría un humano. Pero rara vez toda la información con la que
se cuenta es 100\% acertada, lo que le da en la torre al enfoque puramente
determinista de la inteligencia.

\subsection{Agentes}
En su concepción más tradicional, la inteligencia artificial gira en torno a
\textbf{agentes racionales} que, a través de \textbf{sensores} pueden percibir
su \textbf{entorno} y actuar sobre él a partir de un sistema de decisión.
Internamente, el agente es una máquina compuesta por un conjunto (finito) de estados,
cuyas transiciones están dadas por reglas de inferencias. Cuando una se da una
transición de estados, entonces es realizada una acción. Esta concepción del agente
nos permite reducir la IA en un problema de búsqueda de las funciones de transición.

Computacionalmente hablando, la búsqueda de estas funciones en ciertos problemas de IA
se vuelve absurdamente compleja, por lo que no es posible hacer la codificación de todos
los posibles escenarios en el sistema de inferencia de un agente. Por lo tanto un agente
debe ser capáz de manejar la incertidumbre en su entorno. Además, debe poder \textit{aprender}
del entorno y adaptarse a él, generando conocimiento a través de la experiencia.

El \textbf{aprendizaje de máquina} (\textit{machine learning} en inglés) gira en torno a
algorítmos capáces de manejar información estocástica y generar modelos de toma de decisiones
a partir de ella. Un modelo es bueno en medida de lo acertado que es en términos estadísticos.

\subsection{Tipos de aprendizaje}
El aprendizaje de máquina se ha divido en varios subcampos,
cada uno atacando un tipo de problema distinto y utilizando diferentes
tipos de aprendizaje.
Se describen a continuación cuatro diferentes parámetros bajo los cuáles
es posible clasificar los distintos paradigmas de aprendizaje.

\textbf{Supervisado/no supervisado} Dado que el aprendizaje involucra
una interacción entre el agente y el ambiente, es posible
dividir el aprendizaje con respecto a la naturaleza de dicha interacción.
La primera distinción a notar es la diferencia entre aprendizaje supervisado
y no supervisado. En abstracto, viendo el aprendizaje como un proceso
de ``usar la experiencia para ganar \textit{maestría}'', el aprendizaje
supervisado describe un escenario en el que la ``experiencia'', un
ejemplo de entrenamiento, contiene información significativa que no
está contenida en los ``ejemplos de prueba'' que el sistema aún no ha
visto, y en los que se busca aplicar la \textit{maestría} adquirida.
En esta configuración, la \textit{maestría} adquirida busca
predecir la información faltante de los datos de prueba. Podemos
pensar en el ambiente como un profesor que ``supervisa'' al agente
al proporcionarle información extra (una clasificación). Por otro lado,
en el aprendizaje no supervisado no hay distinción entre datos de
entrenamiento y de prueba. El agente procesa los datos de entrada
con el objetivo de generar una síntesis o versión comprimida de los
datos que sea representativa.

\textbf{Agentes pasivos/activos} En los paradigmas de aprendizaje
el rol tomado por el agente varia. Es posible distinguir entre
agente activo y pasivo. Un agente activo interactua
con el ambiente durante el entrenamiento, realizando consultas o
experimentos, mientras que un agente pasivo sólo observa la
información provista por el ambiente sin influenciarla
o dirigirla durante el proceso de aprendizaje.

\textbf{Ayuda del profesor} Cuando se piensa en el aprendizaje humano,
el proceso por lo general involucra a un profesor, que
está tratando activamente de proveer al agente con la información más útil
para lograr el objetivo de aprendizaje. En contraste, cuando un científico
aprende sobre la naturaleza, el ambiente, tomando el rol de profesor,
puede ser pensado como pasivo: las manzanas caen, las estrellas
brillan y la lluvia cae sin cuidado por el aprendizaje del agente.
Dichos escenarios son modelados postulando que los datos de aprendizaje
(o la experiencia del agente) son generados por un proceso aleatorio.

\textbf{Protocolo de entrenamiento lineal (o en línea)/por lote} El
último parámetro a mencionar es la distinción entre las situaciones en
las que el agente tiene que responder de forma líneal, es decir
que para cada ejemplo de entrenamiento tiene que dar una respuesta, y
cuando el agente tiene que mostrar el aprendizaje adquirido
después de tener la oportunidad de procesar grandes cantidades de datos.
