\section{\sout{Stochastic gradient descent}}
Si se considera el caso en que se tiene un \sout{very large dataset} con millones
de puntos con datos, correr un entrenamiento con \sout{gradient batch descent} puede ser
un proceso sumamente costoso computacionalmente ya que se requiere reevaluar
todo el DATASET cada vez que se toma un \textit{paso} hacia el mínimo global.

Una alternativa popular al algoritmo \sout{gradient batch descent} es \textit{\sout{STOCHASTIC
  GRADIENT DESCENT}}, llamado también \sout{GRADIENT DESCENT} \textit{iterativo}. En lugar
de actualizar los pesos basado en la suma de los errores acumulados de todas las
muestras $x^{(i)}$:
\begin{equation}
  \Delta w = \eta \sum_i(y^{(i)} - \phi(z^{(i)}))x^{(i)}
\end{equation}

Se actualizan los datos de manera incremental para cada muestra del entrenamiento:
\begin{equation}
  \eta(y^{(i)} - \phi(z^{(i)}))x^{(i)}
\end{equation}

Aunque el \sout{STOCHASTIC GRADIENT DESCENT} podria ser considerado una aproximación
del \sout{GRADIENT DESCENT}, por lo general converge mucho más rápido debido a las
actualizaciones tan frecuentes de los pesos. Como cada gradiente se calcula
basado en un sólo ejemplo de entrenamiento, \sout{the error surface is noisier than in gradient descent, which can also have
the advantage that stochastic gradient descent can escape shallow local minima more
readily}. Para obtener resultados \sout{precisos} con \sout{stochastic gradient descent}
es importante que se tomen los datos de forma aleatoria.

Otra ventaja del \sout{stochastic gradient descent} es que se puede usar para
hacer \textit{aprendizaje en línea}. Esto quiere decir que el modelo es entrenado
\sout{on the fly} al momento mientras más y más datos van llegando. Esto es
especialmente útil cuando se están acumulando grandes cantidades de datos.
Usando entrenamiento en línea, el sistema puede adaptarse inmediatamente a
los cambios y los datos de entrenamiento pueden ser descartados después de
actualizar el modelo, si el espacio de almacenamiento fuera un problema.

