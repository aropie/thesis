\sout{Se tomó una base de datos cristalográfica de acoplamientos ya
realizados y sobre esos ligandos y proteinas se corrió Docking con
AutoDockVina n.m. Después se hace una tabla comparando los scores que
les asigna AutoDockVina con el RMSDI cristalográfico. Pero los scores
de autodockvina no siempre son muy buenos y pasa que muchas veces la
que es la mejor pose la manda al $4^o$ o $5^o$ lugar del ranking.}

\section{Deep-pose}
\textbf{Deep-pose} es una red neuronal convolucional profunda que toma
la información de la pose de un acoplamiento en un complejo
proteína-ligando como entrada y produce una calificación de qué tan
\sout{viable} es dicha pose.  Primero, dada una entrada de un complejo
proteína-ligando $x$, se extrae información del contexto local de cada
rama. El \textit{contexto} de una rama está dado por información
estructural básica (\textit{tipo de rama}, y distancia). Después cada
una de estas características básicas de cada rama es convertida en
vectores característicos que utiliza la red para aprender. Luego, una
capa convolucional es empleada para sintetizar la información de todos
los contextos de todas las ramas del ligando y genera una
representación vectorial del complejo. Después se pasa a dos capas
ocultas para sintetización y procesamiento del
vector-ligando. Finalmente, en la última capa, la representación del
complejo es dada como entrada a un clasificador \textit{softmax},
quien es responsable de producir el puntaje. A continuación se
presenta un pseudo-código de alto nivel del proceso de la red:
\begin{algorithm}[H]
  \caption{Deep-pose}
  \begin{algorithmic}[1]
    \State \textbf{Entrada:} complejo proteína-ligando x,
    donde el ligando tiene $n$ \textit{ramas}
    \State \textbf{Dados:}\newline
                           $W^{b\_type} \in \mathbb{R}^{h\times |B|}, W^{b\_dist}
                           \in \mathbb{R}^{h\times |B|},\newline
                           W^{conv} \in \mathbb{R}^{|z_i| \times cf}, W^3 \in
                           \mathbb{R}^{cf \times h},\newline
                           W_{out} \in \mathbb{R}^{h \times 2},\newline
                           b^{conv} \in \mathbb{R}^{cf}, b^{3} \in
                           \mathbb{R}^{h},\newline
                           b^{out} \in \mathbb{R}^{2}$
    \State $Z \gets [..]$
    \For{$i\gets 1, n$}
      \State $z_{b\_type} \gets$ columnas de $W_{b\_type}$
      correspondientes a los tipos de ramas de los vecinos de la
      rama $i$
      \State $z_{b\_dist} \gets$  columnas de $W_{b\_dist}$
      correspondientes a las distancias de los vecinos de la rama $i$
      \State $z_i \gets \{z_{b\_type}, z_{b\_dist}$\}
      \State $Z.add(z_i)$
    \EndFor
    \State // $U$ es inicializada con ceros
    \State $U \gets [..] \in \mathbb{R}^{cf \times n}$
    \State // Capa convolucional
    \For{$i\gets1, n$}
      \State $U[:,i]\gets f(W^{conv}Z[i] + b^{conv})$
    \EndFor
    \State // \textit{max-pooling} por columnas
    \State $r\gets \max(U, axis=1)$
    \State //Capas oculta y de salida
    \State $score\gets W^{out}(W^3r + b^3) + b^{out}$
    \State // Regresa el puntaje normalizado
    \State \textbf{return} $\frac{e^{score[1]}}{e^{score[0]}+e^{score[1]}}$
  \end{algorithmic}
\end{algorithm}
Veamos a detalle cada una de las partes de la red.

\subsection{Contexto de la rama}
Pereira, Caffaren y dos Santos \cite{dossantos} consideran al átomo
como una entidad ligada íntimamente a su contexto. Bajo esta premisa,
crean una red que toma como entrada a cada átomo del ligando con su
contexto codificado, entendiendo contexto del átomo como las
carácteristicas de este y de los átomos más cercanos.
Pereira buscaba encontrar el acoplamiento que generara la mayor
cantidad de energía, sin importar cual fuera la conformación necesaria
para poder realizarlo; nuestro enfoque busca precisamente encontrar
dicha conformación.

Partiendo de la ídea del átomo ligado a su contexto, y considerando
que lo que buscamos encontrar es una propiedad puramente estructural,
tomamos como unidad básica del ligando a los segmentos con libertad
rotacional, a la que llamaremos \textit{rama}. Así, nuestro
\textit{contexto de átomo} se convierte en \textbf{contexto de rama},
siendo este la combinación de el tipo de rama, los de las ramas más
cercanas y la distancia a cada una de ellas.

IMAGEN HISTOGRAMA DE DISTRIBUCIÖN DE TIPOS DE RAMA

Se divide cada receptor y ligando en sus respectivas ramas y cada una
de estas ramas se codifica usando la respresentación
SMILES \footnote{\url{http://www.daylight.com/smiles/}}. Esta
codificación, representa de forma única a cada rama distinta; es a
esto a lo que llamamos \textbf{tipo de rama}. Se enlistan todos los
tipos de rama, asociando a cada uno un índice, generando así lo que
llamamos el \textit{diccionario de ramas}.

Del mismo modo, se segmentan los rangos de distancia encontrados en
compartimentos, y a cada uno de estos se les asigna también un índice,
generando así un \textit{diccionario de distancias}.

\begin{table}[H]
  \begin{center}
    \begin{tabular}{l|l}
      SMILES                 & Id \\ \hline
      NC1=N{[}C{]}(=NC=C1)=O & 93 \\
      C1CCCCC1               & 94 \\
      CNC=O                  & 95 \\
      NC=N                   & 96 \\
      CC=C                   & 97
    \end{tabular}
    \begin{tabular}{l|l}
      Rango de distancia (\AA) & Idx \\ \hline
      3.0526 - 3.2631        & 6   \\
      3.2632 - 3.4736        & 7   \\
      3.4737 - 3.6842        & 8   \\
      3.6843 - 3.8947        & 9   \\
      3.8948 - 4.1052        & 10
    \end{tabular}
  \end{center}
  \caption{Fragmento de los diccionarios de ramas y de distancias}
  \label{fig:dictionary}
\end{table}

A partir de estos diccionarios, se asocia a una rama del ligando con
las cinco ramas del receptor más cercanas codificadas a través de sus
tipos y sus distancias a las ramas dadas. Lo que se genera entonces es
un vector con dos tuplas, donde cada elemento de las tuplas son
índices de tipos de rama y distancia respectivamente; a este vector le
llamamos el \textit{vector de rama}. El conjunto de vectores de rama
de un ligando genera la \textit{matriz de ligando}, que será la
entrada de la red.

\begin{table}[H]
  \begin{center}
  \begin{tabular}{l|l}
    OP(O)O & \AA \\ \hline N & 5.794664 \\ C1CC1 & 5.691862
    \\ NC1=N{[}C{]}(=NC=C1)=O & 4.449922 \\ NC=N & 3.785496 \\ O &
    3.747894
  \end{tabular}
  \end{center}
  \begin{equation*}
  \downarrow
  \end{equation*}
  \begin{equation*}
    OP(O)O=\begin{bmatrix}
    (2, 13, 93, 96, 4) & (4, 2, 2, 6, 4)
    \end{bmatrix}
  \end{equation*}
  \caption{Traducción de una rama a su representación vectorial}
\end{table}

Tanto para los tipos de rama, como para las particiones de distancia, se
generan las matrices $W^{b\_type}$ y $W^{b\_dist}$ respectivamente. Estas
matrices constituyen los pesos de la primera capa de la red y son
inicializadas con valores aleatorios.

\subsection{Representación vectorial del contexto de rama}

La primera capa de la red, toma cada matriz de ligando y la transforma
en una matriz en $\mathbb{R}$. Cada columna en $W^{b\_type} \in
\mathbb{R}^{h\times |B|}$ corresponde al vector característico de un
tipo de rama, donde $B$ es el conjunto de tipo de ramas y $h$ es la
dimensión de los vectores característicos, quedando como un
hiperparámetro a definir.  Dado el contexto de una rama, la red
transforma cada tipo de rama en su respectivo vector característico,
utilizando los índices ya generados, y luego concatena los vectores
para generar a la representación vectorial del tipo de rama
$z_{b\_type}$. Analogamente, se genera el vector $z_{b\_dist}$ en el
contexto de la rama objetivo.

\begin{table}[H]
  \begin{equation*}
    Rama=\begin{bmatrix}
    OP=O, & OC=O, & C=O, & OPO, & OP=O
    \end{bmatrix}
  \end{equation*}
  \begin{center}
    $W_{b\_type}=$
    \begin{tabular}{|l|l|l|l|l|l|l|}
      \hline
      \rowcolor[HTML]{D0D0D0}
      OP=O & OC=O & OPO & C=O & NC=O & OPO & OP=O \\ \hline
      &      &     &     &      &     &      \\ \hline
      &      &     &     &      &     &      \\ \hline
      &      &     &     &      &     &      \\ \hline
    \end{tabular}
  \end{center}
  \begin{equation*}
  \downarrow
  \end{equation*}
  \begin{equation*}
    z_{b\_type}^T =
  \begin{tabular}{lllllllllllllll}
\multicolumn{3}{l}{OP=O}                                               &                       & \multicolumn{3}{l}{OC=O}                                              &                       & \multicolumn{3}{l}{C=O}                                               &                       & \multicolumn{3}{l}{OPO}                                               \\ \cline{1-3} \cline{5-7} \cline{9-11} \cline{13-15}
\multicolumn{1}{|l|}{} & \multicolumn{1}{l|}{} & \multicolumn{1}{l|}{} & \multicolumn{1}{l|}{$\bullet$} & \multicolumn{1}{l|}{} & \multicolumn{1}{l|}{} & \multicolumn{1}{l|}{} & \multicolumn{1}{l|}{$\bullet$} & \multicolumn{1}{l|}{} & \multicolumn{1}{l|}{} & \multicolumn{1}{l|}{} & \multicolumn{1}{l|}{$\bullet$} & \multicolumn{1}{l|}{} & \multicolumn{1}{l|}{} & \multicolumn{1}{l|}{} \\ \cline{1-3} \cline{5-7} \cline{9-11} \cline{13-15}
  \end{tabular}
  \end{equation*}
  \caption{Representación de la construcción del tipo de rama ($z_{b\_type}$).
    El símbolo $\bullet$ representa la operación de concatenación.}
\label{my-label}
\end{table}

Finalmente, la representación del contexto de la rama $b$ se define
como $z_b = z_{b\_type} \cdot z_{b\_dist}$. La idea es que a partir de
características básicas contextuales, la red pueda aprender a
distinguir rasgos abtractos de las ramas que permitan la
discriminación entre poses válidas y señuelos.

\subsection{Representación de la pose de un complejo proteína-ligando}

La segunda capa en la red es una capa convolucional encargada de
extraer más características abstractas de las representaciones de los
contextos de ramas y sintetizar la información de todas ellas en un
vector $r$ de longitud fija.  La preocupación subyaciente que ataca el
uso de una capa convolucional es la capacidad de manejar entradas de
distintas dimensiones. La cantidad de ramas por ligando es variable,
entonces la capa convolucional nos permite procesar complejos de
distintos tamaños.

Dado un complejo $x$ conformado por $n$ ramas, la entrada de la capa
convolucional es una lista de vectores $\{z_1, z_2, ..., z_n\}$ donde
$z_i$ es la representación vectorial del contexto de la $i$-ésima rama
del ligando. En la primera etapa de la capa, la extracción de
características más abstractas de cada vector $z_i$ está dada por
\begin{equation}
  u_i = f(W^{conv}_{z_i} + b^{conv})
\end{equation}
donde $W^{conv} \in \mathbb{R}^{cf \times h_1}$ es la matriz de pesos
correspondiente a la capa convolucional, $b^{conv}$ es el sesgo, $f$ es la
función tangente hiperbólica y $u_i \in \mathbb{R}^{cf}$. El número de
unidades o filtros en la capa convolucional ($cf$) es un hiperparámetro
definido por el usuario.

La segunda etapa en la capa convolucional correspondiende a la etapa
de agrupación (\textit{pooling} en inglés) se encarga de sintetizar
las características de los contextos de rama. La entrada consiste en
un conjunto de vectores $\{u_1, u_2, ..., u_n\}$. Utilizamos una capa
de \textit{max-pooling}, que produce un vector $r \in \mathbb{R}^{cf}$,
donde el valor del $j$-ésimo elemento está definido como el máximo de
los $j$-ésimos elementos del conjunto de vectores de entrada:
\begin{equation}
  [r]_j = \max_{1 \leq i \leq n} [u_i]_j
\end{equation}
El vector resultante $r$ de esta etapa es la representación de la pose
del complejo proteína-ligando. De este modo, la red puede aprender a
generar una representación vectorial que sintetice la información del
complejo que sea relevante para discriminar poses válidas de señuelos.

\subsection{Calificación de la pose}
El vector $r$ es procesado por dos capas neuronales básicas más: una
tercera capa oculta que representa un nivel más de abstracción y una
cuarta y última capa de salida, que computa una calificación para cada
una de las posibles clasificaciones de la pose: (0) pose señuelo y (1)
pose válida. Formalmente, dada la representación $r$ de la pose del
complejo x, la capa oculta y la de salida se computan de la siguiente
manera:
\begin{equation}
  s(x) = W^{out}(w^3r + b^3) + b^{out}
\end{equation}
donde $W^3 \in \mathbb{R}^{hxcf}$ es la matriz de pesos de la tercera
capa oculta $W^{out} \in \mathbb{R}^{2xh}$ es la matriz de pesos de la
capa de salida, y $b^3 \in \mathbb{R}^h$ y $b^{out} \in \mathbb{R}^2$ son
sesgos. El número de unidades en la capa oculta, $h$, es un hiperparámetro
que está definido por el usuario. $s(x) \in \mathbb{R}^2$ es un vector
que contiene las calificaciones de cada una de las dos clases.

Sean $s(x)_0$ y $s(x)_1$ las calificaciones de las clases 0 y 1
respectivamente. Transformamos estas calificaciones en una
distribución de probabilidad utilizando la función \textit{softmax}:
\begin{equation}
  p(0|x) = \frac{e^{s(x)_0}}{e^{s(x)_0}+e^{s(x)_1}}
\end{equation}
\begin{equation}
  p(1|x) = \frac{e^{s(x)_1}}{e^{s(x)_0}+e^{s(x)_1}}
\end{equation}
donde decimos que $p(0|x)$ y $p(1|x)$ es la probabilidad condicional
de que la pose del compuesto sea válida o sea señuelo respectivamente,
dados los datos obtenidos a partir del acoplamiento del complejo
proteína-ligando.

\subsection{Entrenamiento de la red}
\textit{Deep-pose} se entrenó utilizando un algoritmo de
\textbf{descenso estocástico por el gradiente} (SDG por sus siglas en
inglés). En este caso, SDG se utiliza para minimizar la función de
costos sobre un conjunto de entrenamiento $D$ que contiene poses tanto
válidas como señuelos. En cada iteración, un nuevo complejo $(x,y) \in D$
es elegido al azar, donde $y=1$ si la pose es válida y $y=0$ en caso
contrario. Después la red, junto con los parámetros
$\theta = \{W^{b_{type}}, W^{b_{dist}}, W^{conv}, W^3, W^{out}, b^{conv}, b^{3}, b^{out}\}$
es utilizada para estimar la probabilidad $p(y|x, \theta)$. Finalmente,
el error en la predicción se computa como la probabilidad logarítmica
negativa, $-\log(p(y|x, \theta))$, y los parámetros de $\theta$ son
actualizados utilizando \sout{backpropagation}.
\begin{equation}
  \theta \longmapsto \sum_{(x,y) \in D} -\log p(y|x, \theta)
\end{equation}

En este trabajo, se utilizaron \textit{minilotes} de complejos, 30 por
lote, y tomamos el promedio del error de la predicción para realizar
el \sout{backpropagation}. La red se implementó utilizando la biblioteca
Theano\footnote{http://deeplearning.net/software/theano/}.
