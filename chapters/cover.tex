\thispagestyle{empty}
\begin{minipage}{.3\textwidth}
  \flushleft
  \center{\includegraphics[scale=.09]{unam}}

  \vspace{20pt}

  \center{
    \rule{.5pt}{.6\textheight}
    \hspace{7pt}
    \rule{2pt}{.6\textheight}
    \hspace{7pt}
    \rule{.5pt}{.6\textheight}
  } \\

  \center{\includegraphics[scale=.22]{ciencias}}
\end{minipage}
\begin{minipage}{.7\textwidth}
\flushright

\center{

  \center{
    \LARGE{U}\large{NIVERSIDAD} \LARGE{N}\large{ACIONAL} 
    \LARGE{A}\large{UTÓNOMA} \\[10pt]
    \large{DE} 
    \LARGE{M}\large{ÉXICO} 
  } \\
  \rule{\textwidth}{2pt}
  \\
  \hrulefill\\[1cm]
  
  \LARGE{F}\large{ACULTAD DE } \LARGE{C}\large{IENCIAS}\\[2cm]

  \large{
Implementación de redes neuronales convolucionales para el estudio de interacciones proteina-ligando  }\\[2cm]

  \huge{
T \hspace{1cm} E \hspace{1cm} S \hspace{1cm} I \hspace{1cm} S  }\\[1cm]

  \large{QUE PARA OBTENER EL TÍTULO DE:}\\[1cm]

  \large{
Matemático  }\\[1cm]

  \large{PRESENTA:}\\[1cm]

  \large{
Adrián Antonio Rodríguez Pié  }\\[1cm]

  \large{
TUTOR  }\\[1cm]

  \large{
Marcelino Arciniega Castro  }
}\\
  \large{
Ciudad Universitaria, CD. MX., 2018
}

\end{minipage}
