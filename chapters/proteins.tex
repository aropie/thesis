\section{Proteínas}
\cite{tamar}
El término \textbf{proteína} se origina del griego \textit{proteios},
que significa ``primario'' o ``de primer orden''. El nombre fue
adoptado por Jöns Berzelius en 1838 para enfatizar la importancia de
esta clase de moléculas. Las proteínas juegan un rol crucial en el
mantenimiento de la vida (\sout{life-sustaining}). Las proteínas proveen el
soporte para la arquitectura de tejido musculoso, ligamentos, tendones, huesos,
piel, cabello, órganos y glándulas. Las proteínas también proveen los servicios
fundamentales de transporte y almacenamiento como en el caso del oxígeno y
hierro en células musculares y eritrocitos. Las proteínas también juegan un rol
crucial en muchos procesos regulatorios esenciales para la vida, como reacciones
de catálisis (e.g. digestión); funciones inmunológicas y hormonales; y la
coordinación de actividades neuronales, crecimiento de células y hueso, y
diferenciación celular.


\subsection{Acoplamiento molecular}
Basado en \cite{kukol} El campo del \textbf{acoplamiento molecular} o
\textbf{docking} surge a lo largo de las últimas tres décadas gracias a la
necesidad de la biología molecular estructural y el descubrimiento de
hinibidores basado en estructuras. Ha podido evolucionar considerablemente
gracias al crecimiento dramático de disponibilidad y poder de las computadoras,
y al creciente acceso a bases de datos de proteínas y moléculas.

\begin{figure}[H]
  \includegraphics[scale=0.3]{docking} \centering
  \caption{Representación esquemática del \textit{docking}.  (Tomado de
    \url{https://en.wikipedia.org/wiki/Docking_(molecular)})}
\end{figure}

El objetivo de un programa de acoplamiento molecular automatizado es comprender
y predecir reconocimiento molecular, tanto estructuralmente, encontrando
posibles \textit{poses} de acoplamiento, como energéticamente, prediciendo la
afinidad del enlace. El acoplamiento molecular usualmente se realiza entre una
molécula pequeña, llamada ligando, y una macromolécula objetivo, una proteína en
nuestro caso.

\begin{figure}[H]
  \includegraphics[scale=0.5]{docking_steps}
  \caption{Diagrama de flujo para un acoplamiento usual.  (Tomado de
    \cite{kukol})}
  \label{fig:docking_flowchart}
\end{figure}

La figura \ref{fig:docking_flowchart} muestra los pasos clave que son comunes en
todos los protocolos. El acoplamiento consiste en encontrar las poses de unión
más favorables de un ligando hacia una proteína objetivo. La
pose de unión de un ligando puede ser caracterizado de forma única por sus
variables de estado. Estas consisten en su posición (traslaciones sobre los ejes
$x, y, z$), orientación (ángulos de Euler o cuaterniones) y, si el ligando es
flexible, su conformación (los ángulos de torción para cada enlace de
rotación). Cada una de las variables de estado describe un grado de libertad en
un espacio de búsqueda multidimensional.

\subsection{Función evaluadora}
Todos los métodos de acoplamiento requieren una función de evaluación
para calificar las poses de unión de los candidatos, y un método
de búsqueda para explorar las configuraciones de las variables de
estado. En general, el éxito de un acoplamiento se mide en términos de
la \textit{desviación media cuadrática} (RMSD) de las coordenadas
cartesianas de los átomos del ligando en las conformaciones del
acoplamiento, comparadas con las cristalográficas; un acoplamiento se
considera exitoso si el RMSD es menor a 2\AA.

\subsection{Archivos PDB}
El banco de datos de proteínas es un archivo de estructuras de macromoléculas
biológicas determinadas experimentalmente. El formato utilizado para almacenar
está información contiene elementos como coordenadas de átomos, nombres de
moléculas e información sobre estructuras primarias y secundarias. Es con este
formato con el que se trabajó durante el proyecto.
\footnote{\url{ftp://ftp.wwpdb.org/pub/pdb/doc/format_descriptions/Format_v33_Letter.pdf}}

\subsection{SMILES}
SMILES (Simple Molecular Input Line Entry System) es un sencillo lenguaje
químico que permite describir moléculas y reacciones utilizando únicamente
caracteres ASCII que representan símbolos de átomos y enlaces. Una cadena SMILES
contiene la misma información que una tabla de conexiones extendida, pero con
varias ventajas: es sumamente compacta y puede ser canonizada de tal manera
que puede ser usada como identificador universal para una estructura química
dada.\footnote{\url{http://www.daylight.com/smiles/}}
