En este trabajo, se crea \textit{Deep-pose}: una red de aprendizaje
profundo que busca hacer una evaluación de acoplamientos virtuales.
Utilizando \textit{Deep-pose} sobre la salida de acoplamientos
virtuales generados por AutoDock Vina, conseguimos un resultado del
82\% de precisión al momento de discernir entre poses cercanas a las
cristalográficas. Este resultado, aunado a que (1) \textit{Deep-pose}
no requiere características definidas por un humano, a diferencia del
proceso \textit{artesanal} que se sigue usualmente para determinar
cuáles acoplamientos son buenos, y que (2) alcanza buenos resultados a
partir de la salida de un sólo programa de docking, hacen que
\textit{Deep-pose} sea atractivo para usarse como complemento al
acoplamiento virtual usado normalmente. Además, este trabajo propone
ideas novedosas sobre cómo modelar complejos proteínas-ligandos e
introduce una propuesta que demuestra ser efectiva para codificar
configuraciones moleculares, que podría ser usada para otros fines en
alguna otra red de aprendizaje profundo aplicada a bioinformática.
