\section{Abstract}
El descubrimiento de nuevos fármacos es un proceso muy tardado y
costoso; e incluso el reposicionamiento de compuestos ya conocidos es
una tarea sumamente complicada. El escenario es aun más complejo si
se toman en cuenta los miles o millones de moléculas capaces de ser
sintetizadas en cada etapa del desarrollo.  Para superar estas
dificultades, el uso de alternativas computacionales de bajo costo es
altamente recomendado y se ha adoptado como la forma estándar de
ayuda para el desarrollo de nuevos fármacos.

Una de las metodologias computacionales más usadas para investigar las
interacciones proteína-ligando es el acoplamiento molecular. La
selección de los ligandos con mayor afinidad utilizando Cribado
Virtual basado en Acoplamiento (DBVS, \textit{Docking Based Virtual
  Screening}) es realizada mediante la inserción de cada elemento de
la bibilioteca en una región particular de un receptor objetivo.

En la primera etapa del proceso se procede a una búsqueda heurística
en la que miles de posibles inserciones son consideradas. En la
segunda etapa, la calidad de la inserción es calificada a través de
una función evaluadora. Esta última fase se ha convertido en todo un
reto para los científicos computacionales, por la dificultad de
decidir de forma determinística si un acoplamiento es bueno o no.

Los sistemas basados en aprendizaje de máquina (ML, \textit{Machine
  Learning}) han sido usados con éxito para mejorar la salida del DBVS
para incrementar el desempeño de las funciones evaluadoras así como
para construir clasificadores de afinidad de enlace. Una de las
principales ventajas de utilizar ML es la capacidad de explicar la
dependencia no lineal de las interacciones moleculares entre ligando y
receptor.

En este trabajo se propone un acercamiento con una red neuronal
convolucional para mejorar el DBVS. El método utiliza los resultados
de una simulación de acoplamiento como entrada, en donde
automáticamente aprende a extraer características relevantes a partir
de datos básicos como tipos de átomo, distancias entre ellos, y su
contexto en la rama estructural. La red aprende características
abstractas que son útiles para discriminar entre poses válidas y
señuelos en un acoplamiento.
