\section{Workflow}
\subsection{¿Qué se hizo?}
Se tomó una base de datos cristalográfica de acoplamientos ya
realizados y sobre esos ligandos y proteinas se corrió Docking con
AutoDockVina n.m. Después se hace una tabla comparando los scores que
les asigna AutoDockVina con el RMSDI cristalográfico. Pero los scores
de autodockvina no siempre son muy buenos y pasa que muchas veces la
que es la mejor pose la manda al $4^o$ o $5^o$ lugar del ranking.
Aquí es donde entra la red:

DosSantos \cite{dossantos} crea una red de metaanálisis de los scores
del docking, tomando como entradas a cada átomo del ligando con su
contexto (entendido como las características de los átomos aledaños).
Partiendo de esta idea, y considerando al docking como un proceso
puramente estructural, tomamos como unidad básica a las ramas de cada
ligando encapsulamos a cada una con la información de las ramas más
cercanas.

Esta infromación consiste en tipo de rama y distancia. Para el tipo
de rama, se hace un listado con todos las ramas distintas presentes en
la base de datos (apróx. 5000) y se le asocia un número a cada una.

\subsection{¿Cómo se hizo?}
Se toma una base de datos cristalográfica y se hace docking sobre
ella. En paralelo, se toman archvios PDB y sus correspondientes PDBQT
y se generan objetos de Python donde cada objeto representa ya sea un
ligando o una proteína y se asocia a cada uno su respectivo score y
RMSDI. Después en otro script, se procesa cada ligando y se genera
acorde a la entrada de la red.
\subsubsection{Contexto de la rama}
Para extraer la información principal de la rama, se hace primero un
índice (o diccionario) con todos los tipos de ramas distintas. A
partir de ese diccionario (con su respectivo índice), se asocia a una
rama dada las cinco ramas más cercanas codificadas a través de sus
tipos y sus distancias a las ramas dadas. Esto se hace para cada rama
del ligando y el conjunto de estas constituye la codificación del ligando.
\subsection{¿Para qué?}
