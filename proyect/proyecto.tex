\documentclass{article}

\usepackage[utf8]{inputenc}
\usepackage{geometry}
\geometry{margin=1in, tmargin=0.2in}


\renewcommand\refname{Bibliografía}

\begin{document}
\title{Implementación de redes neuronales convolucionales para el estudio de interacciones proteína-ligando}
\author{Adrián Antonio Rodríguez Pié}
\date{}

\maketitle

El descubrimiento de nuevos fármacos es un proceso muy tardado y costoso.
El desarrollo e incluso el reposicionamiento de compuestos ya conocidos es una
tarea sumamente complicada. El escenario es aún más complicado si se toman en cuenta
los miles o millones de moléculas capáces de ser sintetizadas en cada etapa del desarrollo.
Para superar estas dificultades, el uso de alternativas computacionales de bajo costo
es altamente recomendado, y se ha adoptado como la forma estándar te ayuda para el
desarrollo de nuevos fármacos.

Una de las metodologias computacionales más usades para investigar estas interacciones es
el acoplamiento molecular. La selección de los ligandos más potentes utilizando Cribado
Virtual basado en Acoplamiento (DBVS) es realizado a través de realizar la inserción de
cada compuesto de la bibilioteca de compuestos en una región particular de un receptor
objetivo. En la primera etapa del proceso, una búsqueda heurística es llevada a cabo
en la que miles de posibles inserciones son consideradas. En la segunda etapa, la calidad
de la inserción es evaluada a través de una función evaluadora. Esta última fase se ha
convertido en todo un reto para los científicos computacionales, por la dificultad de decidir
de forma determinística si un acoplamiento es bueno o no.

Los sistemas basados en aprendizaje de máquina (ML) han sido usados con éxito para mejorar la
salida del DBVS para tanto incrementar el desempeño de las funciones evaluadoras, como para
construir clasificadores de afinidad de enlace. Una de las principales ventajas de utilizar ML
es la capacidad de explicar la dependencia no lineal de las interacciones moleculares entre
ligando y receptor.

En este trabajo se propone un acercamiento con una red neuronal convolucional para mejorar el
DBVS. El método utiliza los resultados de una simulación de acoplamiento como entrada, en
donde automáticamente aprende a extraer características relevantes a partir de datos básicos
como tipos de átomo, distancias entre ellos, y su contexto en la rama estructural. La red
aprende características abstractas que son útiles para discriminar entre ligandos activos y
señuelos en una proteína.

\begin{thebibliography}{5}
\bibitem{Arciniega}\textsc{Arciniega M, \& Lange OF.}
  Improvement of Virtual Screening Results by Docking Data Feature Analysis.
  \textit{J. Chem. Inf. Model.}  \textbf{2014}; 54:1401-11.

\bibitem{Lounnas}\textsc{Lounnas V, Ritschel T, Kelder J, McGuire R, Bywater RP, \& Foloppe N.}
  Current progress in Structure-Based Rational Drug Design marks a new mindset in drug discovery.
  \textit{Comput. Struct. Biotechnol. J.} \textbf{2013}; 5:e201302011.

\bibitem{Pereira}\textsc{Pereira, J. C., Caffarena, E. R., \& dos Santos, C. N.}
  Boosting docking-based virtual screening with deep learning.
  \textit{Journal of chemical information and modeling} \textbf{2016}; 56:2495-2506.

\bibitem{Bishop}\textsc{Bishop, C.} (2007).
  \textit{Pattern Recognition and Machine Learning (Information Science and Statistics)},
  1a edición. 2006. corr. 2a edición. Springer, New York.

\bibitem{Lu}\textsc{John Lu, Z. Q.} (2010).
  \textit{The elements of statistical learning: data mining, inference, and prediction.}
  Journal of the Royal Statistical Society: Series A (Statistics in Society), 173:693-694.
\end{thebibliography}

\vspace{12mm}
\noindent
\begin{center}
\begin{tabular}{ll}
\makebox[2.5in]{\hrulefill} & \makebox[2.5in]{\hrulefill}\\
Tesista & Asesor\\[8ex]
\end{tabular}
\end{center}

\end{document}
